\documentclass[a4paper,12pt]{article}

\usepackage{appendixnumberbeamer}
\usepackage{booktabs}
\usepackage[scale=2]{ccicons}
\usepackage{pgfplots}
\usepackage{xspace}
\usepackage{bookmark}
\usepackage{amssymb}
\usepackage{mathtools}
\usepackage[normalem]{ulem}
\usepackage[T1]{fontenc}
\usepackage[sfdefault,book]{FiraSans}
\usepackage{FiraMono}
\usepackage{hyperref}
\usetheme{metropolis}

\usepgfplotslibrary{dateplot}

\lstset{
  basicstyle=\ttfamily,
  escapeinside={\%*}{*)},
  mathescape=true,
  literate=
  {á}{{\'a}}1 {é}{{\'e}}1 {í}{{\'i}}1 {ó}{{\'o}}1 {ú}{{\'u}}1
  {Á}{{\'A}}1 {É}{{\'E}}1 {Í}{{\'I}}1 {Ó}{{\'O}}1 {Ú}{{\'U}}1
  {à}{{\`a}}1 {è}{{\`e}}1 {ì}{{\`i}}1 {ò}{{\`o}}1 {ù}{{\`u}}1
  {À}{{\`A}}1 {È}{{\'E}}1 {Ì}{{\`I}}1 {Ò}{{\`O}}1 {Ù}{{\`U}}1
  {ä}{{\"a}}1 {ë}{{\"e}}1 {ï}{{\"i}}1 {ö}{{\"o}}1 {ü}{{\"u}}1
  {Ä}{{\"A}}1 {Ë}{{\"E}}1 {Ï}{{\"I}}1 {Ö}{{\"O}}1 {Ü}{{\"U}}1
  {â}{{\^a}}1 {ê}{{\^e}}1 {î}{{\^i}}1 {ô}{{\^o}}1 {û}{{\^u}}1
  {Â}{{\^A}}1 {Ê}{{\^E}}1 {Î}{{\^I}}1 {Ô}{{\^O}}1 {Û}{{\^U}}1
  {Ã}{{\~A}}1 {ã}{{\~a}}1 {Õ}{{\~O}}1 {õ}{{\~o}}1
  {œ}{{\oe}}1 {Œ}{{\OE}}1 {æ}{{\ae}}1 {Æ}{{\AE}}1 {ß}{{\ss}}1
  {ű}{{\H{u}}}1 {Ű}{{\H{U}}}1 {ő}{{\H{o}}}1 {Ő}{{\H{O}}}1
  {ç}{{\c c}}1 {Ç}{{\c C}}1 {ø}{{\o}}1 {å}{{\r a}}1 {Å}{{\r A}}1
  {€}{{\euro}}1 {£}{{\pounds}}1 {«}{{\guillemotleft}}1
  {»}{{\guillemotright}}1 {ñ}{{\~n}}1 {Ñ}{{\~N}}1 {¿}{{?`}}1
}


\title{Desenvolvimento de Software para a Web II -- Roteiro 1}
\author{Prof. Thiago Cavalcante}
\date{}

\begin{document}

\pagenumbering{gobble}
\maketitle

\sloppy
\raggedright

\begin{enumerate}
  \item Instale o editor de texto\footnote{Olhar o \textbf{tutorial}}
  \item Instale os programas necessários\footnotemark[1]
  \item Crie a pasta de projetos\footnotemark[1]
  \item Crie um app Rails chamado \textbf{primeiro\_app} \\
        usando o comando \textbf{rails <versão> new <nome\_app>}\footnotemark[1]
  \item Atualize o arquivo \textbf{Gemfile} e reinstalar os \\
        pacotes com o \textbf{Bundler}\footnotemark[1]
  \item Veja o app funcionando com o comando \\
        \textbf{rails server}%
        \footnote[7]{Esperar aparecer a mensagem sobre \textbf{Ctrl-C}}
  \item \textbf{Exercícios}

    \begin{enumerate}
      \item Verifique a \textbf{versão do Ruby} na página \\
            inicial do seu app e confirme rodando o \\
            comando \textbf{ruby -v} no terminal
      \item Verifique a \textbf{versão do Rails} na página \\
            inicial do seu app; verifique que é a \\
            mesma versão que consta no \textbf{Gemfile}
    \end{enumerate}

  \item Crie uma ação \textbf{ola} no controlador da \\
        aplicação

    \begin{lstlisting}[language=Ruby, title=app/controllers/application\_controller.rb]
class ApplicationController < ActionController::Base
  def ola
    render html: "olá, mundo!"
  end
end
    \end{lstlisting}

  \pagebreak

  \item Adicione uma rota para direcionar a ação \\
        anterior para a página inicial do app

    \begin{lstlisting}[language=Ruby, title=config/routes.rb]
Rails.application.routes.draw do
  root 'application#ola'
end
    \end{lstlisting}

  \item \textbf{Exercícios}

    \begin{enumerate}
      \item Modifique a ação \textbf{ola} para exibir a \\
            mensagem \textbf{"olá, pessoal!"}
      \item Crie uma segunda ação chamada \textbf{tchau} \\
            que exiba o texto \textbf{"tchau, mundo!"} e \\
            modifique a rota da página inicial \\
            para esta nova ação
    \end{enumerate}

  \item Faça a configuração inicial do \textbf{git}

    \begin{lstlisting}[language=Bash, basicstyle=\fontsize{8.6}{12}\selectfont\ttfamily]
git config --global user.name "Seu Nome"
git config --global user.email seu.email@email.com
git config --global credential.helper "cache --timeout=86400"
    \end{lstlisting}

  \item Adicione todos os arquivos ao repositório e \\
        faça o primeiro \textbf{commit}

    \begin{lstlisting}[language=Bash, commentstyle=\color{gray}]
git status # checando arquivos antes de adicionar
git add -A
git status # checando arquivos depois de adicionar
git commit -m "Inicializar repositório"
git log # checando histórico de commits
    \end{lstlisting}

  \item Crie um \textbf{token} de acesso pessoal no GitHub, de acordo com o tutorial desse link (pode dar o nome que quiser ao token, por exemplo "token-rails"):

  \textcolor{red}{\textbf{\footnotesize https://help.github.com/pt/github/authenticating-to-github/creating-a-personal-access-token-for-the-command-line}}

  \item Crie um repositório no \textbf{GitHub}, associe-o \\
        ao seu repositório local e envie o código

    \begin{lstlisting}[language=Bash]
git remote add origin <link https>
git push -u origin master
    \end{lstlisting}

  \textcolor{red}{\textbf{IMPORTANTE: Quando for pedida a senha no terminal, use o token gerado no passo anterior e não a senha que você usa para logar no site do GitHub}}

  \item Crie um novo \textbf{branch} no repositório para \\
        alterar o \textbf{README} com a descrição do projeto

    \begin{lstlisting}[language=Bash, commentstyle=\color{gray}]
git checkout -b modificar-README
git branch # checando o branch atual
git status # checando as modificações atuais
git commit -a -m "Melhorar o README"
    \end{lstlisting}

  \item Insira as modificações do \textbf{branch} criado no \\
        \textbf{branch} principal do projeto (\textbf{master})

    \begin{lstlisting}[language=Bash, commentstyle=\color{gray}]
git checkout master
git merge modificar-README
git branch -d modificar-README # deletando o branch
git push # enviando o código para o GitHub
    \end{lstlisting}

  \item Adicione a \textbf{gem} do \textbf{PostgreSQL} no \textbf{Gemfile}, no \\
        ambiente de produção, para assegurar \\
        compatibilidade com o \textbf{Heroku}

    \begin{lstlisting}[language=Ruby, title=Gemfile]
group :production do
  gem 'pg', '0.20.0'
end
    \end{lstlisting}

  \item Modifique o \textbf{Gemfile} de forma que a \textbf{gem} do \\
        \textbf{SQLite} seja usada apenas nos ambientes de \\
        desenvolvimento e teste

    \begin{lstlisting}[language=Ruby, title=Gemfile]
group :development, :test do
  gem 'sqlite3', '1.4.1'
  ...
end
    \end{lstlisting}

  \item Atualize os pacotes com o \textbf{Bundler} e salve as \\
        alterações no \textbf{git}

    \begin{lstlisting}[language=Bash, commentstyle=\color{gray}]
bundle update
bundle install --without production
git commit -a -m "Atualizar Gemfile para o Heroku"
    \end{lstlisting}

  \item Faça login no \textbf{Heroku} e crie o app

    \begin{lstlisting}[language=Bash, commentstyle=\color{gray}]
heroku login --interactive
heroku create # dentro da pasta do projeto
    \end{lstlisting}

  \item Envie o código para o \textbf{Heroku}

    \begin{lstlisting}[language=Bash, commentstyle=\color{gray}]
git push heroku master
    \end{lstlisting}

  \item \textbf{Exercícios}

    \begin{enumerate}
      \item repita os mesmos exercícios do tópico \\
            \textbf{10}, dessa vez fazendo o \textbf{deployment} \\
            para o \textbf{Heroku}
      \item Rode \textbf{heroku help} para ver a lista de \\
            comandos do \textbf{Heroku} e descubra qual \\
            deles exibe os registros da atividade \\
            do seu app (\textbf{logs})
      \item Use o comando anterior para ver o \\
            \textbf{evento mais recente}
    \end{enumerate}
\end{enumerate}

\end{document}
