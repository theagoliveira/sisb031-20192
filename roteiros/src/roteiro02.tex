\documentclass[a4paper,12pt]{article}

\usepackage{appendixnumberbeamer}
\usepackage{booktabs}
\usepackage[scale=2]{ccicons}
\usepackage{pgfplots}
\usepackage{xspace}
\usepackage{bookmark}
\usepackage{amssymb}
\usepackage{mathtools}
\usepackage[normalem]{ulem}
\usepackage[T1]{fontenc}
\usepackage[sfdefault,book]{FiraSans}
\usepackage{FiraMono}
\usepackage{hyperref}
\usetheme{metropolis}

\usepgfplotslibrary{dateplot}

\lstset{
  basicstyle=\ttfamily,
  escapeinside={\%*}{*)},
  mathescape=true,
  literate=
  {á}{{\'a}}1 {é}{{\'e}}1 {í}{{\'i}}1 {ó}{{\'o}}1 {ú}{{\'u}}1
  {Á}{{\'A}}1 {É}{{\'E}}1 {Í}{{\'I}}1 {Ó}{{\'O}}1 {Ú}{{\'U}}1
  {à}{{\`a}}1 {è}{{\`e}}1 {ì}{{\`i}}1 {ò}{{\`o}}1 {ù}{{\`u}}1
  {À}{{\`A}}1 {È}{{\'E}}1 {Ì}{{\`I}}1 {Ò}{{\`O}}1 {Ù}{{\`U}}1
  {ä}{{\"a}}1 {ë}{{\"e}}1 {ï}{{\"i}}1 {ö}{{\"o}}1 {ü}{{\"u}}1
  {Ä}{{\"A}}1 {Ë}{{\"E}}1 {Ï}{{\"I}}1 {Ö}{{\"O}}1 {Ü}{{\"U}}1
  {â}{{\^a}}1 {ê}{{\^e}}1 {î}{{\^i}}1 {ô}{{\^o}}1 {û}{{\^u}}1
  {Â}{{\^A}}1 {Ê}{{\^E}}1 {Î}{{\^I}}1 {Ô}{{\^O}}1 {Û}{{\^U}}1
  {Ã}{{\~A}}1 {ã}{{\~a}}1 {Õ}{{\~O}}1 {õ}{{\~o}}1
  {œ}{{\oe}}1 {Œ}{{\OE}}1 {æ}{{\ae}}1 {Æ}{{\AE}}1 {ß}{{\ss}}1
  {ű}{{\H{u}}}1 {Ű}{{\H{U}}}1 {ő}{{\H{o}}}1 {Ő}{{\H{O}}}1
  {ç}{{\c c}}1 {Ç}{{\c C}}1 {ø}{{\o}}1 {å}{{\r a}}1 {Å}{{\r A}}1
  {€}{{\euro}}1 {£}{{\pounds}}1 {«}{{\guillemotleft}}1
  {»}{{\guillemotright}}1 {ñ}{{\~n}}1 {Ñ}{{\~N}}1 {¿}{{?`}}1
}


\title{Desenvolvimento de Software para a Web II -- Roteiro 2}
\author{Prof. Thiago Cavalcante}
\date{}

\begin{document}

\pagenumbering{gobble}
\maketitle

\begin{enumerate}
  \item Crie um app Rails chamado \textbf{segundo\_app}
  \item Substitua o \textbf{Gemfile} padrão pelo Gemfile do \\
        \textbf{primeiro\_app} e atualize os pacotes com o \\
        \textbf{Bundler}\footnote{Lembrar de usar a opção \textbf{--without production}}
  \item Adicione os arquivos no \textbf{git} e faça o \\
        \textbf{primeiro commit}
  \item Crie um repositório no \textbf{GitHub} e envie o \\
        código
  \item De forma semelhante ao \textbf{primeiro\_app}, crie \\
        uma ação \textbf{ola}, modifique a rota da página \\
        inicial, faça um \textbf{commit}, crie o app no \\
        \textbf{Heroku} e envie o código para o \textbf{GitHub} \\
        e para o \textbf{Heroku} (\textit{deployment})
  \item Gere o recurso de \textbf{usuários} com o comando \\
        \textbf{scaffold}

    \begin{lstlisting}[language=Bash, basicstyle=\fontsize{9.8}{12}\selectfont\ttfamily]
rails generate scaffold User name:string email:string
    \end{lstlisting}

  \item Faça a migração do banco de dados

    \begin{lstlisting}[language=Bash]
rails db:migrate
    \end{lstlisting}

  \item Rode o seu app com \textbf{rails server} para \\
        explorar as páginas de usuários

  \item \textbf{Exercícios}

    \begin{enumerate}
      \item Crie um novo usuário e inspecione o \\
            \textbf{código-fonte} da página para descobrir \\
            o \textbf{id} CSS da mensagem de confirmação; o \\
            que acontece quando a página é \\
            atualizada?
      \item O que acontece ao tentar criar um \\
            usuário sem e-mail?
      \item O que acontece ao tentar criar um \\
            usuário com um e-mail inválido?
      \item Destrua os usuários anteriores; o \\
            app mostra alguma mensagem quando \\
            o usuário é destruído?
    \end{enumerate}

  \item Modifique o arquivo de rotas para que a \\
        página inicial do app leve à página inicial \\
        dos usuários

    \begin{lstlisting}[language=Ruby, title=config/routes.rb]
Rails.application.routes.draw do
  resources :users
  root 'users#index'
end
    \end{lstlisting}

  \item Faça um diagrama da arquitetura MVC e \\
        explique os passos que são realizados ao \\
        acessar a página \textbf{/users/1/edit}
  \item Encontre no código a linha que obtém do \\
        banco de dados as informações do usuário do \\
        exercício anterior (dica: \textbf{set\_user})
  \item Qual é o nome do arquivo de visualização \\
        para a página de edição do usuário?

  \item Gere o recurso de \textbf{microposts} com o comando \\
        \textbf{scaffold} e faça a migração do banco

    \begin{lstlisting}[language=Bash, basicstyle=\fontsize{8.5}{12}\selectfont\ttfamily]
rails generate scaffold Micropost content:text user_id:integer
rails db:migrate
    \end{lstlisting}

  \item \textbf{Exercícios}

    \begin{enumerate}
      \item Repita o exercício \textbf{9.(a)} para os \\
            microposts
      \item Tente criar um micropost sem \\
            conteúdo e sem id de usuário
      \item Tente criar um micropost com \\
            mais de 140 caracteres
      \item Destrua os microposts anteriores
    \end{enumerate}

  \item Crie uma \textbf{validação} para o tamanho do \\
        conteúdo no \textbf{modelo} dos microposts

    \begin{lstlisting}[language=Ruby, title=app/models/micropost.rb]
class Micropost < ApplicationRecord
  validates :content, length: { maximum: 140 }
end
    \end{lstlisting}

  \item \textbf{Exercícios}

    \begin{enumerate}
      \item Repita o exercício \textbf{15.(c)}; existe alguma \\
            mudança no resultado?
      \item Inspecione o \textbf{código-fonte} da página para \\
            descobrir o \textbf{id} CSS da mensagem de erro \\
            produzida na questão anterior
    \end{enumerate}

  \item Crie uma \textbf{associação} entre usuários e \\
        microposts

    \begin{lstlisting}[language=Ruby, title=app/models/user.rb]
class User < ApplicationRecord
  has_many :microposts
end
    \end{lstlisting}

    \begin{lstlisting}[language=Ruby, title=app/models/micropost.rb]
class Micropost < ApplicationRecord
  belongs_to :user
  validates :content, length: { maximum: 140 }
end
    \end{lstlisting}

  \item Abra o \textbf{console} do rails para verificar o \\
        resultado da associação usando \textbf{rails console}

  \item \textbf{Exercícios}

    \begin{enumerate}
      \item Edite a página que mostra (\textbf{show}) o \\
            usuário para mostrar também o conteúdo \\
            do seu \textbf{primeiro micropost}
      \item Crie uma validação para a \textbf{presença} do \\
            conteúdo no modelo de micropost \\
            (\textbf{presence: true})
      \item Baseado no exercício anterior, crie \\
            validações para a presença do nome \\
            e e-mail no modelo de usuário
    \end{enumerate}

  \item \textbf{Exercícios}

    \begin{enumerate}
      \item Encontre a linha, no controlador da \\
            aplicação, que mostra que a classe \\
            \textbf{ApplicationController} herda da classe \\
            \textbf{Action-Controller::Base}
      \item Existe um arquivo mostrando que a classe \\
            \textbf{ApplicationRecord} herda da classe \\
            \textbf{ActiveRecord::Base}?
    \end{enumerate}

  \item Faça um commit com as alterações e envie \\
        para o GitHub e Heroku (deu tudo certo?)

  \item Use \textbf{heroku logs} para checar o erro no \\
        \textit{deployment}

  \item Faça a migração do banco de dados no \\
        \textbf{Heroku} com \textbf{heroku run rails db:migrate}
\end{enumerate}
\end{document}
