\documentclass[a4paper,12pt]{article}

\usepackage{appendixnumberbeamer}
\usepackage{booktabs}
\usepackage[scale=2]{ccicons}
\usepackage{pgfplots}
\usepackage{xspace}
\usepackage{bookmark}
\usepackage{amssymb}
\usepackage{mathtools}
\usepackage[normalem]{ulem}
\usepackage[T1]{fontenc}
\usepackage[sfdefault,book]{FiraSans}
\usepackage{FiraMono}
\usepackage{hyperref}
\usetheme{metropolis}

\usepgfplotslibrary{dateplot}

\lstset{
  basicstyle=\ttfamily,
  escapeinside={\%*}{*)},
  mathescape=true,
  literate=
  {á}{{\'a}}1 {é}{{\'e}}1 {í}{{\'i}}1 {ó}{{\'o}}1 {ú}{{\'u}}1
  {Á}{{\'A}}1 {É}{{\'E}}1 {Í}{{\'I}}1 {Ó}{{\'O}}1 {Ú}{{\'U}}1
  {à}{{\`a}}1 {è}{{\`e}}1 {ì}{{\`i}}1 {ò}{{\`o}}1 {ù}{{\`u}}1
  {À}{{\`A}}1 {È}{{\'E}}1 {Ì}{{\`I}}1 {Ò}{{\`O}}1 {Ù}{{\`U}}1
  {ä}{{\"a}}1 {ë}{{\"e}}1 {ï}{{\"i}}1 {ö}{{\"o}}1 {ü}{{\"u}}1
  {Ä}{{\"A}}1 {Ë}{{\"E}}1 {Ï}{{\"I}}1 {Ö}{{\"O}}1 {Ü}{{\"U}}1
  {â}{{\^a}}1 {ê}{{\^e}}1 {î}{{\^i}}1 {ô}{{\^o}}1 {û}{{\^u}}1
  {Â}{{\^A}}1 {Ê}{{\^E}}1 {Î}{{\^I}}1 {Ô}{{\^O}}1 {Û}{{\^U}}1
  {Ã}{{\~A}}1 {ã}{{\~a}}1 {Õ}{{\~O}}1 {õ}{{\~o}}1
  {œ}{{\oe}}1 {Œ}{{\OE}}1 {æ}{{\ae}}1 {Æ}{{\AE}}1 {ß}{{\ss}}1
  {ű}{{\H{u}}}1 {Ű}{{\H{U}}}1 {ő}{{\H{o}}}1 {Ő}{{\H{O}}}1
  {ç}{{\c c}}1 {Ç}{{\c C}}1 {ø}{{\o}}1 {å}{{\r a}}1 {Å}{{\r A}}1
  {€}{{\euro}}1 {£}{{\pounds}}1 {«}{{\guillemotleft}}1
  {»}{{\guillemotright}}1 {ñ}{{\~n}}1 {Ñ}{{\~N}}1 {¿}{{?`}}1
}


\title{Desenvolvimento de Software para a Web II -- Roteiro 2}
\author{Prof. Thiago Cavalcante}
\date{}

\begin{document}

\pagenumbering{gobble}
\maketitle

\sloppy
\raggedright

\begin{enumerate}
  \item Crie um app Rails chamado \textbf{segundo\_app} com o comando \textbf{rails new segundo\_app}
  \item Substitua o conteúdo do \textbf{Gemfile} do segundo\_app com o conteúdo do Gemfile do \textbf{primeiro\_app} e atualize os pacotes com o \textbf{Bundler} usando \textbf{bundle update} e \textbf{bundle install --without production}
  \item Adicione os arquivos no \textbf{git} com \textbf{git add -A} e faça o \textbf{primeiro commit} com \textbf{git commit -m "Inicializar o repositório"}
  \item Crie um novo repositório chamado \textbf{segundo\_app} no \textbf{GitHub} \textcolor{red}{(lembre-se de enviar o repositório para mim ou me adicionar como colaborador)} e envie o código do seu repositório local para ele
  \item De forma semelhante ao \textbf{primeiro\_app}, crie uma ação \textbf{ola} no controlador da aplicação, modifique a rota da página inicial para essa ação e \textbf{faça um commit} para salvar essas alterações
  \item Crie o app no \textbf{Heroku} com \textbf{heroku create} \textcolor{red}{(lembre-se de enviar o link do Heroku para mim)} e envie o código para o \textbf{GitHub} e para o \textbf{Heroku} (\textit{deployment}) usando \textbf{git push origin master} e \textbf{git push heroku master}
  \item Crie o recurso de \textbf{usuários} com o comando \textbf{scaffold}, isso vai gerar o código para implementar usuários com nome e e-mail no seu app

    \begin{lstlisting}[language=Bash, basicstyle=\fontsize{9.8}{12}\selectfont\ttfamily]
rails generate scaffold User name:string email:string
    \end{lstlisting}

  \item Faça a migração do banco de dados, ou seja, crie a tabela de usuários no banco, com o comando abaixo

    \begin{lstlisting}[language=Bash]
rails db:migrate
    \end{lstlisting}

  \item Rode o seu app com \textbf{rails server} para explorar as páginas de usuários. Crie, edite e remova usuários. Clique na página de um usuário para ver suas informações. Observe a URL que aparece na barra de endereços quando você faz cada uma dessas atividades. O comando \textbf{scaffold} gera uma série de ações e visualizações associadas, de acordo com a tabela abaixo:
  \begin{center}
    \begin{tabular}{@{}lll@{}}
      \toprule
      \textbf{URL} & \textbf{Ação} & \textbf{Descrição} \\
      \midrule
      /users & \textbf{index} & Lista todos os usuários \\
      /users/1 & \textbf{show} & Mostra o usuário com id 1 \\
      /users/new & \textbf{new} & Cria um novo usuário \\
      /users/1/edit & \textbf{edit} & Edita o usuário com id 1 \\
      \bottomrule
    \end{tabular}
  \end{center}

  \item \textbf{Exercícios}

    \begin{enumerate}
      \item Crie um novo usuário e inspecione o \textbf{código-fonte} da página para descobrir o \textbf{id} CSS da mensagem de confirmação. Qual é o id? O que acontece quando a página é atualizada? \textcolor{red}{(responder por escrito)}
      \item O que acontece ao tentar criar um usuário sem e-mail? \textcolor{red}{(responder por escrito)}
      \item O que acontece ao tentar criar um usuário com um e-mail inválido? \textcolor{red}{(responder por escrito)}
      \item Destrua os usuários das questões anteriores. O app mostra alguma mensagem quando o usuário é destruído? \textcolor{red}{(responder por escrito)}
    \end{enumerate}

  \item Modifique o arquivo de rotas para que a página inicial do app leve à página inicial dos usuários

    \begin{lstlisting}[language=Ruby, title=config/routes.rb]
Rails.application.routes.draw do
  resources :users
  root 'users#index'
end
    \end{lstlisting}

  \item Abra no editor de texto o arquivo do controlador de usuário (\textbf{app/controllers/users\_controller.rb}). Veja as ações (métodos) que estão definidas lá. Quais são? \textcolor{red}{(responder por escrito)}

  % \item Faça um diagrama da arquitetura MVC e explique os passos que são realizados ao acessar a página \textbf{/users/1/edit}
  % \item Encontre no código a linha que obtém do banco de dados as informações do usuário do exercício anterior (dica: \textbf{set\_user})
  % \item Qual é o nome do arquivo de visualização para a página de edição do usuário?

  \item De forma semelhante aos passos 7 e 8, crie o recurso de \textbf{microposts} com o comando \textbf{scaffold} e faça a migração do banco. Isso vai gerar o código para implementar microposts com conteúdo e id de usuário (autor do micropost)

    \begin{lstlisting}[language=Bash, basicstyle=\fontsize{8.5}{12}\selectfont\ttfamily]
rails generate scaffold Micropost content:text user_id:integer
rails db:migrate
    \end{lstlisting}

  \item \textbf{Exercícios}

    \begin{enumerate}
      \item Repita o exercício \textbf{10.(a)} para os microposts \textcolor{red}{(responder por escrito)}
      \item Tente criar um micropost sem conteúdo e sem id de usuário. O que acontece? \textcolor{red}{(responder por escrito)}
      \item Tente criar um micropost com mais de 140 caracteres. O que acontece? \textcolor{red}{(responder por escrito)}
      \item Destrua os microposts das questões anteriores
    \end{enumerate}

  \item Abra no editor de texto o arquivo do controlador de microposts (\textbf{app/controllers/microposts\_controller.rb}). Veja as ações (métodos) que estão definidas lá. Quais são? \textcolor{red}{(responder por escrito)}

  \item Crie uma \textbf{validação} para o tamanho do conteúdo no \textbf{modelo} dos microposts (verifique que deu certo tentando criar um micropost com mais de 140 caracteres)

    \begin{lstlisting}[language=Ruby, title=app/models/micropost.rb]
class Micropost < ApplicationRecord
  validates :content, length: { maximum: 140 }
end
    \end{lstlisting}

  \item \textbf{Exercícios}

    \begin{enumerate}
      \item Repita o exercício \textbf{14.(c)}; existe alguma mudança no resultado? \textcolor{red}{(responder por escrito)}
      \item Inspecione o \textbf{código-fonte} da página para descobrir o \textbf{id} CSS da mensagem de erro produzida na questão anterior. Qual o id? \textcolor{red}{(responder por escrito)}
    \end{enumerate}

  \item Crie uma \textbf{associação} entre usuários e microposts usando \textbf{has\_many} e \textbf{belongs\_to}. Dessa forma, o aplicativo vai poder associar os microposts aos usuários que os criaram, usando o campo \textbf{user\_id} presente na tabela de microposts.

    \begin{lstlisting}[language=Ruby, title=app/models/user.rb]
class User < ApplicationRecord
  has_many :microposts
end
    \end{lstlisting}

    \begin{lstlisting}[language=Ruby, title=app/models/micropost.rb]
class Micropost < ApplicationRecord
  belongs_to :user
  validates :content, length: { maximum: 140 }
end
    \end{lstlisting}

  \item Para verificar o resultado da associaçao que foi criada, abra o console do rails rodando o comando \textbf{rails console} no seu terminal. Ao abrir o terminal, atribua a uma variável \textbf{primeiro\_usuario} o valor \textbf{User.first} (ou seja, ela recebe o primeiro usuário guardado no banco de dados) e rode o comando \textbf{primeiro\_usuario.microposts}. Com isso, serão exibidos todos os microposts associados ao primeiro usuário. Para verificar o caminho contrário da associação (começando a partir de um micropost), execute os comandos a seguir. \textcolor{red}{(obs.: crie pelo menos um usuário e vários microposts com o id desse usuário, para poder ver os resultados)}

  \begin{lstlisting}[language=Ruby, title=app/models/micropost.rb]
algum_micropost = primeiro_usuario.microposts.first
algum_micropost.user
  \end{lstlisting}

  \item \textbf{Exercícios}

    \begin{enumerate}
      \item Edite a página que mostra (\textbf{show}) o usuário para mostrar também o conteúdo do seu \textbf{primeiro micropost}
      \item Baseado no passo \textbf{16} do roteiro, crie uma validação para a \textbf{presença} do \textbf{conteúdo} no \textbf{modelo de micropost} (\textbf{DICA: "presence: true"}) (verifique que deu certo)
      \item Baseado no exercício anterior, crie também validações para a \textbf{presença} do \textbf{nome} e \textbf{e-mail} no \textbf{modelo de usuário} (verifique que deu certo)
    \end{enumerate}

  \item \textbf{Exercícios}

    \begin{enumerate}
      \item Encontre a linha, no controlador da aplicação, que mostra que a classe \textbf{ApplicationController} herda da classe \textbf{Action-Controller::Base} \textcolor{red}{(responder por escrito)}
      \item Existe um arquivo mostrando que a classe \textbf{ApplicationRecord} herda da classe \textbf{ActiveRecord::Base}? Qual? (dica: procure na pasta de modelos) \textcolor{red}{(responder por escrito)}
    \end{enumerate}

  \item Faça um commit com as alterações (exemplo de mensagem: "Finalizar segundo\_app") e envie para o GitHub e Heroku usando o comando git push. Quando você entra na página do Heroku do seu app, o site funciona normalmente? \textcolor{red}{(responder por escrito)}

  \item Use \textbf{heroku logs} para checar o erro no \textit{deployment} (procure por ActionView::Template::Error). Qual a mensagem de erro completa? \textcolor{red}{(responder por escrito)}

  \item Faça a migração do banco de dados no \textbf{Heroku} com \textbf{heroku run rails db:migrate} e acesse a página do seu app no Heroku novamente e veja que ela está funcionando
\end{enumerate}
\end{document}
