\documentclass[a4paper,10pt]{article}

\usepackage[utf8]{inputenc}
\usepackage[T1]{fontenc}
\usepackage{alphalph}
\usepackage[shortlabels]{enumitem}
\usepackage{caption}
\usepackage{amssymb}
\usepackage{amsmath}
\usepackage{fontawesome}
\usepackage{geometry}
\usepackage[brazil]{babel}
\usepackage{listings}
\usepackage{color}
\usepackage{xcolor}
\usepackage{plex-mono}
\usepackage[sfdefault]{plex-sans}
\usepackage{booktabs}
\usepackage{graphicx}
\usepackage{ragged2e}
\usepackage[none]{hyphenat}
\usepackage[outputdir={../pdf/build_pdflatex}]{minted}
\usepackage{multicol}
\usepackage{pgfplots}
\usepackage{ulem}
\usepackage{tikz}
\usepackage{hyperref}
\usepackage{textcomp}
\usepackage{blkarray}
\usepackage{titling}

\renewcommand{\thefootnote}{\fnsymbol{footnote}}
\renewcommand{\MintedPygmentize}{/home/thiago/.local/bin/pygmentize}

\definecolor{light-gray}{gray}{0.95}
\renewcommand\lstlistingname{Código}
\DeclareCaptionFormat{listing} {
  \parbox{\textwidth}{\hspace{-0.2cm}#1#2#3}
}
\DeclareCaptionFont{black}{\color{black}}
\captionsetup[lstlisting]{
  format=listing,
  labelfont=black,
  textfont=black,
  singlelinecheck=true,
  margin=0pt,
  font={tt,footnotesize,bf}
}

\lstset{
  basicstyle=\footnotesize\ttfamily,
  escapeinside={\%*}{*)},
  mathescape=true,
  showspaces=false,
  showtabs=false,
  showstringspaces=false,%
  rulesepcolor=\color{black},
  frame=shadowbox,
  upquote=true,
  literate=
  {á}{{\'a}}1 {é}{{\'e}}1 {í}{{\'i}}1 {ó}{{\'o}}1 {ú}{{\'u}}1
  {Á}{{\'A}}1 {É}{{\'E}}1 {Í}{{\'I}}1 {Ó}{{\'O}}1 {Ú}{{\'U}}1
  {à}{{\`a}}1 {è}{{\`e}}1 {ì}{{\`i}}1 {ò}{{\`o}}1 {ù}{{\`u}}1
  {À}{{\`A}}1 {È}{{\'E}}1 {Ì}{{\`I}}1 {Ò}{{\`O}}1 {Ù}{{\`U}}1
  {ä}{{\"a}}1 {ë}{{\"e}}1 {ï}{{\"i}}1 {ö}{{\"o}}1 {ü}{{\"u}}1
  {Ä}{{\"A}}1 {Ë}{{\"E}}1 {Ï}{{\"I}}1 {Ö}{{\"O}}1 {Ü}{{\"U}}1
  {â}{{\^a}}1 {ê}{{\^e}}1 {î}{{\^i}}1 {ô}{{\^o}}1 {û}{{\^u}}1
  {Â}{{\^A}}1 {Ê}{{\^E}}1 {Î}{{\^I}}1 {Ô}{{\^O}}1 {Û}{{\^U}}1
  {Ã}{{\~A}}1 {ã}{{\~a}}1 {Õ}{{\~O}}1 {õ}{{\~o}}1
  {œ}{{\oe}}1 {Œ}{{\OE}}1 {æ}{{\ae}}1 {Æ}{{\AE}}1 {ß}{{\ss}}1
  {ű}{{\H{u}}}1 {Ű}{{\H{U}}}1 {ő}{{\H{o}}}1 {Ő}{{\H{O}}}1
  {ç}{{\c c}}1 {Ç}{{\c C}}1 {ø}{{\o}}1 {å}{{\r a}}1 {Å}{{\r A}}1
  {€}{{\euro}}1 {£}{{\pounds}}1 {«}{{\guillemotleft}}1
  {»}{{\guillemotright}}1 {ñ}{{\~n}}1 {Ñ}{{\~N}}1 {¿}{{?`}}1
}

\setminted{
  fontsize=\footnotesize,
  style=bw,
  frame=single,
  labelposition=topline,
}

\pretitle{\begin{center}\normalsize\bfseries\MakeUppercase{Desenvolvimento de Software para a Web II -- }\MakeUppercase}
\author{\normalsize Prof. Thiago Cavalcante}
\date{\vspace{-4ex}}

\geometry{
  top=0.0cm,
  left=1.0cm,
  right=1.0cm,
  bottom=1.0cm
}

\setlength\columnsep{30pt}

\usetikzlibrary{arrows}

\makeatletter
\AddEnumerateCounter{\PaddingUp}{\two@digits}{A00}
\AddEnumerateCounter{\PaddingDown}{\two@digits}{A00}
\newcommand\PaddingUp[1]{\expandafter\two@digits\csname c@#1\endcsname}
\newcommand\PaddingDown[1]{\PaddingUp{#1}\addtocounter{#1}{-2}}
\makeatother

\makeatletter
\def\enumalphalphcnt#1{\expandafter\@enumalphalphcnt\csname c@#1\endcsname}
\def\@enumalphalphcnt#1{\AlphAlph{#1}}
\makeatother
\AddEnumerateCounter{\enumalphalphcnt}{\@enumalphalphcnt}{aa}

\pagenumbering{gobble}


\title{Final}
\posttitle{\end{center}}

\begin{document}

\maketitle

\emergencystretch 3em

NOME: \rule{.85\textwidth}{0.1mm}

\bigskip

Relacione as colunas a seguir:

\bigskip

\begin{minipage}[c]{0.7\linewidth}
  \setlength{\leftmargini}{0pt}
  \begin{enumerate}[label=(\PaddingUp*),itemsep=.1em]
    \item Método usado para associar um símbolo a um valor em uma visualização
    \item Parte da arquitetura MVC que fornece páginas HTML
    \item Uma das ações de controlador gerada pelo \textit{scaffold}
    \item Nome dado a uma condição imposta aos atributos de um modelo
    \item Envio do projeto Rails para ser rodado no Heroku
    \item Arquivo de um projeto Rails que especifica os pacotes e suas versões
    \item Comando para abrir o terminal de linha de comando do projeto Rails
    \item Comando do Heroku para criar um novo app na plataforma
    \item Comando do git para criar um novo \textit{branch} ou mudar para um já existente
    \item Nome dado ao repositório remoto do GitHub
    \item Nome do arquivo de layout de um aplicativo Rails
    \item Tag da estrutura padrão de uma página HTML que contém o seu conteúdo
    \item Método usado para criar uma tag HTML com um link para outra página
    \item Extensão de um arquivo de visualização de um projeto Rails
    \item Pasta de um projeto Rails onde fica armazenado o arquivo do roteador
    \item Comando do git usado para enviar o código para um repositório remoto
    \item Parte da arquitetura MVC que acessa o banco de dados
    \item Método usado para criar uma tag HTML de imagem
    \item Comando do Heroku para entrar com o usuário e a senha no serviço
    \item Comando do layout da aplicação que carrega a visualização desejada
    \item Método de teste que verifica o resultado do carregamento de uma página
    \item Uma das operações do protocolo HTTP
    \item Comando do bash para mudar de pasta
    \item Comando para rodar todos os testes de um projeto Rails
    \item Nome da pasta que armazena as visualizações do projeto Rails
    \item Extensão de um arquivo de programa escrito em Ruby
    \item Nome dado à rota da página inicial do aplicativo Rails
    \item Comando do git para salvar mudanças no histórico
    \item Comando usado para rodar um aplicativo Rails localmente
    \item Comando do git que inclui as modificações de um \textit{branch} em outro \textit{branch}
    \item Um dos comandos que cria associações entre modelos em um projeto Rails
    \item Comando usado para gerar código automaticamente em um projeto Rails
    \item Tag da estrutura padrão de uma página HTML que contém o seu título
    \item Método de teste que verifica o conteúdo de uma tag HTML
    \item Dada a existência de um controlador \textit{Usuario}, nome do seu arquivo de testes
    \item Nome dado ao repositório remoto do Heroku
    \item Comando usado para criar ou alterar tabelas no banco de dados
    \item Pasta do projeto Rails que armazena imagens e arquivos de estilo
    \item Endereço padrão usado para acessar um aplicativo Rails localmente
    \item Framework CSS
  \end{enumerate}
\end{minipage} % no space if you would like to put them side by side
\begin{minipage}[c]{0.2\linewidth}
  \setlength{\leftmargini}{0pt}
  \begin{enumerate}[label=(\textcolor{white}{\PaddingUp*}),itemsep=.1em]
    \item body
    \item rb
    \item show
    \item server
    \item cd
    \item usuario\_controller\_test
    \item heroku
    \item origin
    \item assert\_response
    \item deployment
    \item db:migrate
    \item provide
    \item create
    \item config
    \item login
    \item image\_tag
    \item push
    \item link\_to
    \item assert\_select
    \item Validação
    \item Modelo
    \item views
    \item checkout
    \item has\_many
    \item yield
    \item html.erb
    \item test
    \item root
    \item assets
    \item commit
    \item Gemfile
    \item Visualização
    \item localhost:3000
    \item application.html.erb
    \item merge
    \item head
    \item generate
    \item console
    \item Bootstrap
    \item GET
    % RESPOSTA:
    % 12
    % 26
    % 03
    % 29
    % 23
    % 35
    % 36
    % 10
    % 21
    % 05
    % 37
    % 01
    % 08
    % 15
    % 19
    % 18
    % 16
    % 13
    % 34
    % 04
    % 17
    % 25
    % 09
    % 31
    % 20
    % 14
    % 24
    % 27
    % 38
    % 28
    % 06
    % 02
    % 39
    % 11
    % 30
    % 33
    % 32
    % 07
    % 40
    % 22
  \end{enumerate}
\end{minipage}
\end{document}
