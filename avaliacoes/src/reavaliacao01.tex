\documentclass[a4paper,10pt]{article}

\usepackage{appendixnumberbeamer}
\usepackage{booktabs}
\usepackage[scale=2]{ccicons}
\usepackage{pgfplots}
\usepackage{xspace}
\usepackage{bookmark}
\usepackage{amssymb}
\usepackage{mathtools}
\usepackage[normalem]{ulem}
\usepackage[T1]{fontenc}
\usepackage[sfdefault,book]{FiraSans}
\usepackage{FiraMono}
\usepackage{hyperref}
\usetheme{metropolis}

\usepgfplotslibrary{dateplot}

\lstset{
  basicstyle=\ttfamily,
  escapeinside={\%*}{*)},
  mathescape=true,
  literate=
  {á}{{\'a}}1 {é}{{\'e}}1 {í}{{\'i}}1 {ó}{{\'o}}1 {ú}{{\'u}}1
  {Á}{{\'A}}1 {É}{{\'E}}1 {Í}{{\'I}}1 {Ó}{{\'O}}1 {Ú}{{\'U}}1
  {à}{{\`a}}1 {è}{{\`e}}1 {ì}{{\`i}}1 {ò}{{\`o}}1 {ù}{{\`u}}1
  {À}{{\`A}}1 {È}{{\'E}}1 {Ì}{{\`I}}1 {Ò}{{\`O}}1 {Ù}{{\`U}}1
  {ä}{{\"a}}1 {ë}{{\"e}}1 {ï}{{\"i}}1 {ö}{{\"o}}1 {ü}{{\"u}}1
  {Ä}{{\"A}}1 {Ë}{{\"E}}1 {Ï}{{\"I}}1 {Ö}{{\"O}}1 {Ü}{{\"U}}1
  {â}{{\^a}}1 {ê}{{\^e}}1 {î}{{\^i}}1 {ô}{{\^o}}1 {û}{{\^u}}1
  {Â}{{\^A}}1 {Ê}{{\^E}}1 {Î}{{\^I}}1 {Ô}{{\^O}}1 {Û}{{\^U}}1
  {Ã}{{\~A}}1 {ã}{{\~a}}1 {Õ}{{\~O}}1 {õ}{{\~o}}1
  {œ}{{\oe}}1 {Œ}{{\OE}}1 {æ}{{\ae}}1 {Æ}{{\AE}}1 {ß}{{\ss}}1
  {ű}{{\H{u}}}1 {Ű}{{\H{U}}}1 {ő}{{\H{o}}}1 {Ő}{{\H{O}}}1
  {ç}{{\c c}}1 {Ç}{{\c C}}1 {ø}{{\o}}1 {å}{{\r a}}1 {Å}{{\r A}}1
  {€}{{\euro}}1 {£}{{\pounds}}1 {«}{{\guillemotleft}}1
  {»}{{\guillemotright}}1 {ñ}{{\~n}}1 {Ñ}{{\~N}}1 {¿}{{?`}}1
}


\title{Reavaliação AB1}
\posttitle{\end{center}}

\begin{document}

\maketitle

\emergencystretch 3em

\begin{itemize}[itemsep=0em]
  \item Não use celular/computador e não converse com ninguém, a prova é individual.
  \item Sinta-se à vontade para tirar dúvidas (\textbf{razoáveis}) ou pedir esclarecimentos sobre as questões.
  \item Use \textbf{letra legível}! não posso dar nota para algo que não consigo ler.
  \item Lembre-se de \textbf{assinar seu nome nas suas folhas}. Se usar \textbf{mais de uma} folha, \textbf{enumere cada página}.
  \item \textbf{Seja organizado:} especifique número e letra da questão que você está respondendo e deixe um espaço entre as respostas, para não ficar tudo amontoado. Você pode pegar mais folhas, se precisar.
\end{itemize}


NOME: \rule{.85\textwidth}{0.1mm}

\begin{multicols*}{2}
\setlength{\leftmargini}{0pt}
\begin{enumerate}
  \item (4,4 pt $\rightarrow$ 11 x 0,4 pt) Sobre o desenvolvimento de aplicativos web com Rails, responda as questões a seguir:

  \begin{enumerate}
    \item Se você deseja criar um aplicativo Rails chamado \textit{app\_rails}, qual comando deve rodar no terminal? % rails new app_rails
    \item Qual o nome do arquivo que contém informações sobre os pacotes Ruby que estão incluídos no projeto? % Gemfile
    \item Qual comando é usado para rodar o aplicativo localmente no seu computador? % rails server
    \item Suponha que o controlador da aplicação contém uma ação chamada \textit{teste\_html} definida em seu código, a qual exibe um texto na tela. Que linha de código deve ser adicionada ao roteador da aplicação para que a página incial do aplicativo seja direcionada para essa ação? % root 'application#teste_html'
    \item Durante a configuração inicial do git para controle de versões do projeto, que informações o desenvolvedor deve fornecer? % nome e email
    \item Escreva os comandos necessários para salvar as modificações do código do projeto no histórico do git. Use a mensagem "Salvar modificações". % git add -A; git commit -m "Salvar modificações"
    \item Sempre que é feita uma adição ou remoção de pacotes no projeto, é preciso rodar dois comandos no terminal para instalar ou excluir os pacotes. Que comandos são esses? % bundle update e bundle install
    \item Qual comando deve ser executado no terminal para a criação de um novo aplicativo na plataforma Heroku? % heroku create
    \item Suponha que a execução do comando da letra anterior resultou na criação de um app chamado \textit{obscure-springs-55859}. Escreva a URL que deve ser usada para acessar o site da aplicação. % https://obscure-springs-55859.herokuapp.com
    \item Qual comando é utilizado para fazer o \textit{deployment} da aplicação para o Heroku? % git push heroku master
    \item Qual comando é utilizado para atualizar o repositório remoto do projeto no GitHub? % git push origin master
  \end{enumerate}

  \item (1,6 pt) A arquitetura de funcionamento do Rails é composta pelos seguintes elementos: navegador de internet (ou \textit{browser}), banco de dados, visualização, controlador, modelo e roteador. Desenhe um diagrama que estabeleça as conexões entre cada elemento. Certifique-se de mostrar as direções de comunicação entre cada parte.

  \bigskip\bigskip

  % SOURCES
  % - https://github.com/kumar91gopi/Algorithms-and-Data-Structures-in-Ruby
  \item (2,0 pt) Leia o código Ruby a seguir e responda as perguntas.

  \begin{minted}{ruby}
def metodo(a,b)
  resultado = 0

  while b > 0
    resultado += a if (b % 2 == 1)
    a *= 2
    b /= 2
  end

  return resultado
end
  \end{minted}

  \begin{enumerate}
    \item (0,6 pt) Qual o resultado desse método para: a = 2 e b = 12? a = 5 e b = 8? % 24, 40
    \item (0,7 pt) Qual é o propósito desse método?
    \item (0,7 pt) O método continua realizando seu propósito se: a = 0? b = 0? a < 0? b < 0? % Sim, sim, sim, não
  \end{enumerate}

  % SOURCES
  % - https://github.com/kumar91gopi/Algorithms-and-Data-Structures-in-Ruby
  \item (2,0 pt) Leia o código Ruby a seguir e responda as perguntas.

  \begin{minted}{ruby}
def metodo(lista,num)
  tam = lista.length

  lista.sort!

  esq = 0
  dir = tam - 1

  while esq < dir
    if (lista[esq] + lista[dir] == num)
      return true
    elsif (lista[esq] + lista[dir] > num)
      dir -= 1
    else
      esq += 1
    end
  end

  return false
end
  \end{minted}

  \begin{enumerate}
    \item (0,6 pt) Qual o resultado desse método para: lista = [11, 5, 6, 3, 1, 8] e num = 10? lista = [1, 5, 7, 3, 2, 8] e num = 7? % false, true
    \item (0,7 pt) Qual é o propósito desse método? % Encontrar um par de números em uma lista cuja soma seja igual a num
    \item (0,7 pt) O método continua realizando seu propósito se a linha \texttt{lista.sort!} for removida? Explique o motivo se a resposta for não. % Não. Com a lista nõo ordenada, o incremento/decremento dos índices esq/dir não garante que a próxima soma vai ser a maior/menor soma que segue a soma anterior.
  \end{enumerate}
\end{enumerate}
\end{multicols*}
\end{document}
