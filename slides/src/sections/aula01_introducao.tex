\begin{frame}[fragile]{Informações Gerais da Disciplina}
    \begin{center}
        \begin{tabular}{@{}ll@{}}
            \toprule
            \textbf{Nome} & Desenvolvimento de Software para a Web II \\
            \textbf{Código} & SISB031 \\
            \textbf{Semestre} & 7º \\
            \textbf{Carga Horária} & 60h \\
            \textbf{PPC} & 03/2018 \\
            \textbf{Turma} & 2019.2 \\
            \textbf{Horário} & Quarta, 19:50 -- 22:30 \\
            \textbf{Sala} & 2 \\
            \textbf{Lista de discussão} & Google Groups: \href{https://groups.google.com/forum/#!forum/sisb031_20192}{\alert{sisb031\_20192}} \\
            \textbf{Repositório} & GitHub: \href{https://github.com/theagoliveira/sisb031_20192}{\alert{theagoliveira/sisb031\_20192}} \\
            \bottomrule
        \end{tabular}
    \end{center}
\end{frame}

\begin{frame}{Cronograma}
    \begin{center}
        \begin{tabular}{@{}r@{/}r@{/}rll@{}}
            \toprule
            \multicolumn{4}{@{}l}{\textbf{Datas importantes}} \\
            \midrule
            21 & 11 & 2019 & \textit{Possível} dia da \alert{AB1} \\
            23 & 11 & 2019 & Prazo final para digitação da \alert{AB1} \\
            13 & 02 & 2020 & \textit{Possível} dia da \alert{AB2} \\
            17 & 02 & 2020 & Prazo final para digitação da \alert{AB2} \\
            17--22 & 02 & 2020 & Período de \alert{reavaliação} \\ 
            27--29 & 02 & 2020 & Período de \alert{provas finais} \\ 
            \bottomrule
        \end{tabular}
    \end{center}
\end{frame}

\begin{frame}{Recapitulando: Desenv. de Software para a Web I}
    \begin{itemize}
        \item Introdução ao ambiente WEB
        \item Projetos de sistemas para web: modelo cliente-servidor, padrão MVC, arquitetura em camadas, protocolo HTTP
        \item Introdução e fundamentação ao HTML: textos, paragráfos, containers, hiperlinks, tabelas, imagens, formulários e menus
        \item Introdução às folhas de estilo em cascata (CSS)
        \item Conceitos de scripts do lado do cliente com Javascript e suas aplicações em páginas web
        \item Introdução a Web dinâmica
        \item Desenv. de aplicações web usando \textit{Server Side Scripts} (PHP)
        \item Gerenciamento de estado de aplicações web
        \item Criação de formulários web
        \item Desenvolvimento de aplicações com banco de dados
    \end{itemize}
\end{frame}

\begin{frame}{Ruby on Rails}
    \includegraphics[width=\textwidth]{images/rails_logo.png}
\end{frame}

\setbeamertemplate{frame footer}{https://ideamotive.co/blog/40-best-ruby-on-rails-companies-websites/}
\begin{frame}{Por que Rails?}
    Top 10 empresas que usam Rails em 2019:
    
    \begin{itemize}
        \item AirBNB
        \item Groupon
        \item GitHub
        \item Couchsurfing
        \item Shopify
        \item Ask.fm
        \item Dribble
        \item Twitter
        \item Etsy
        \item Fab
    \end{itemize}
\end{frame}
\setbeamertemplate{frame footer}{}

\begin{frame}{Bibliografia}
    \huge
    Codecademy: Learn Ruby
    
    \bigskip
    
    Michael Hartl: Ruby on Rails Tutorial
\end{frame}

\begin{frame}{Conceitos de Ruby}
    \Large
    \begin{center}
        \includegraphics[width=0.5\textwidth]{images/ruby_logo.png}
    \end{center}
    
    Link para programar em Ruby no navegador:
    
    https://repl.it/languages/ruby
\end{frame}