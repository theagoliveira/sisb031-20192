\documentclass[10pt]{beamer}

\usepackage[utf8]{inputenc}
\usepackage[T1]{fontenc}
\usepackage{alphalph}
\usepackage[shortlabels]{enumitem}
\usepackage{caption}
\usepackage{amssymb}
\usepackage{amsmath}
\usepackage{fontawesome}
\usepackage{geometry}
\usepackage[brazil]{babel}
\usepackage{listings}
\usepackage{color}
\usepackage{xcolor}
\usepackage{plex-mono}
\usepackage[sfdefault]{plex-sans}
\usepackage{booktabs}
\usepackage{graphicx}
\usepackage{ragged2e}
\usepackage[none]{hyphenat}
\usepackage[outputdir={../pdf/build_pdflatex}]{minted}
\usepackage{multicol}
\usepackage{pgfplots}
\usepackage{ulem}
\usepackage{tikz}
\usepackage{hyperref}
\usepackage{textcomp}
\usepackage{blkarray}
\usepackage{titling}

\renewcommand{\thefootnote}{\fnsymbol{footnote}}
\renewcommand{\MintedPygmentize}{/home/thiago/.local/bin/pygmentize}

\definecolor{light-gray}{gray}{0.95}
\renewcommand\lstlistingname{Código}
\DeclareCaptionFormat{listing} {
  \parbox{\textwidth}{\hspace{-0.2cm}#1#2#3}
}
\DeclareCaptionFont{black}{\color{black}}
\captionsetup[lstlisting]{
  format=listing,
  labelfont=black,
  textfont=black,
  singlelinecheck=true,
  margin=0pt,
  font={tt,footnotesize,bf}
}

\lstset{
  basicstyle=\footnotesize\ttfamily,
  escapeinside={\%*}{*)},
  mathescape=true,
  showspaces=false,
  showtabs=false,
  showstringspaces=false,%
  rulesepcolor=\color{black},
  frame=shadowbox,
  upquote=true,
  literate=
  {á}{{\'a}}1 {é}{{\'e}}1 {í}{{\'i}}1 {ó}{{\'o}}1 {ú}{{\'u}}1
  {Á}{{\'A}}1 {É}{{\'E}}1 {Í}{{\'I}}1 {Ó}{{\'O}}1 {Ú}{{\'U}}1
  {à}{{\`a}}1 {è}{{\`e}}1 {ì}{{\`i}}1 {ò}{{\`o}}1 {ù}{{\`u}}1
  {À}{{\`A}}1 {È}{{\'E}}1 {Ì}{{\`I}}1 {Ò}{{\`O}}1 {Ù}{{\`U}}1
  {ä}{{\"a}}1 {ë}{{\"e}}1 {ï}{{\"i}}1 {ö}{{\"o}}1 {ü}{{\"u}}1
  {Ä}{{\"A}}1 {Ë}{{\"E}}1 {Ï}{{\"I}}1 {Ö}{{\"O}}1 {Ü}{{\"U}}1
  {â}{{\^a}}1 {ê}{{\^e}}1 {î}{{\^i}}1 {ô}{{\^o}}1 {û}{{\^u}}1
  {Â}{{\^A}}1 {Ê}{{\^E}}1 {Î}{{\^I}}1 {Ô}{{\^O}}1 {Û}{{\^U}}1
  {Ã}{{\~A}}1 {ã}{{\~a}}1 {Õ}{{\~O}}1 {õ}{{\~o}}1
  {œ}{{\oe}}1 {Œ}{{\OE}}1 {æ}{{\ae}}1 {Æ}{{\AE}}1 {ß}{{\ss}}1
  {ű}{{\H{u}}}1 {Ű}{{\H{U}}}1 {ő}{{\H{o}}}1 {Ő}{{\H{O}}}1
  {ç}{{\c c}}1 {Ç}{{\c C}}1 {ø}{{\o}}1 {å}{{\r a}}1 {Å}{{\r A}}1
  {€}{{\euro}}1 {£}{{\pounds}}1 {«}{{\guillemotleft}}1
  {»}{{\guillemotright}}1 {ñ}{{\~n}}1 {Ñ}{{\~N}}1 {¿}{{?`}}1
}

\setminted{
  fontsize=\footnotesize,
  style=bw,
  frame=single,
  labelposition=topline,
}

\pretitle{\begin{center}\normalsize\bfseries\MakeUppercase{Desenvolvimento de Software para a Web II -- }\MakeUppercase}
\author{\normalsize Prof. Thiago Cavalcante}
\date{\vspace{-4ex}}

\geometry{
  top=0.0cm,
  left=1.0cm,
  right=1.0cm,
  bottom=1.0cm
}

\setlength\columnsep{30pt}

\usetikzlibrary{arrows}

\makeatletter
\AddEnumerateCounter{\PaddingUp}{\two@digits}{A00}
\AddEnumerateCounter{\PaddingDown}{\two@digits}{A00}
\newcommand\PaddingUp[1]{\expandafter\two@digits\csname c@#1\endcsname}
\newcommand\PaddingDown[1]{\PaddingUp{#1}\addtocounter{#1}{-2}}
\makeatother

\makeatletter
\def\enumalphalphcnt#1{\expandafter\@enumalphalphcnt\csname c@#1\endcsname}
\def\@enumalphalphcnt#1{\AlphAlph{#1}}
\makeatother
\AddEnumerateCounter{\enumalphalphcnt}{\@enumalphalphcnt}{aa}

\pagenumbering{gobble}

\title{Desenvolvimento de Software para a Web II}
\author{Thiago Cavalcante  -- thiago.kun@gmail.com}
\institute{Universidade Federal de Alagoas -- UFAL \\ Campus Arapiraca \\ Unidade de Ensino de Penedo}
\titlegraphic{\hfill\includegraphics[height=1.5cm]{images/brasao-ufal.eps}}

\subtitle{Aula 3}
\date{07 de novembro de 2019}

\begin{document}

\maketitle

\begin{frame}{Introdução ao Rails e Desenvolvimento Web}
  \huge
  \textbf{O que é o Rails?}
  \vfill
  \LARGE
  \textit{Framework} grátis e \textit{open-source} para desenvolvimento \textit{web} escrito em \textbf{Ruby}
\end{frame}

\begin{frame}{Introdução ao Rails e Desenvolvimento Web}
  \huge
  \textbf{E o que é um \textit{framework}??}
  \vfill
  \LARGE
  Conjunto de bibliotecas ou componentes que são usados para \textbf{criar uma base} onde sua aplicação será construída
\end{frame}

\begin{frame}{Introdução ao Rails e Desenvolvimento Web}
  \huge
  \textbf{E o que é um \textit{framework}??}
  \vfill
  \LARGE
  \textit{Frameworks} ajudam no desenvolvimento \textbf{rápido e seguro} de aplicações, mas é recomendável estudar antes a tecnologia em que a mesma é desenvolvida
\end{frame}

\begin{frame}{Introdução ao Rails e Desenvolvimento Web}
  \huge
  \textbf{Vantagens do Rails}
  \vfill
  \LARGE
  \begin{itemize}
    \item \textit{Full-stack}
    \item Integrado
    \item Ótimo \textit{back end} para páginas dinâmicas (ex.: página única)
    \item Estável (existe desde 2004)
  \end{itemize}
\end{frame}

\setbeamertemplate{frame footer}{https://rubyroidlabs.com/blog/2019/10/9-industries-where-flagship-companies-choose-ruby-on-rails/}
\begin{frame}{Introdução ao Rails e Desenvolvimento Web}
  \huge
  \includegraphics[width=\textwidth]{images/rails_companies.jpg}
\end{frame}
\setbeamertemplate{frame footer}{}

\begin{frame}{Introdução ao Rails e Desenvolvimento Web}
  \huge
  \textbf{O que você vai usar em um projeto Rails:}
  \vfill
  \large
  \begin{itemize}
    \item Ruby
    \item Bash (linha de comando do Linux)
    \item HTML
    \item CSS
    \item JavaScript
    \item SQL
    \item Git
  \end{itemize}
\end{frame}

\begin{frame}{Introdução ao Rails e Desenvolvimento Web}
  \huge
  \textbf{Objetivo}
  \vfill
  \Large
  Construir uma aplicação em Rails com algumas características como cadastro, login e posts de usuários. Realizar o controle de versão com o Git e o \textit{deployment} na internet.
\end{frame}

\begin{frame}{Introdução ao Rails e Desenvolvimento Web}
  \huge
  \textbf{Ambiente de desenvolvimento}
  \vfill
  \Large
  \begin{itemize}
    \item Plataforma na nuvem (\textbf{Amazon AWS Cloud9} -- grátis, requer cadastro com cartão de crédito, requer internet)
    \item Instalação no computador (mais complicado, mais aprendizado, pode desenvolver sem internet)
  \end{itemize}
\end{frame}

\begin{frame}{Introdução ao Rails e Desenvolvimento Web}
  \huge
  \textbf{Componentes essenciais}
  \vfill
  \LARGE
  \begin{itemize}
    \item Terminal de linha de comando
    \item Explorador de arquivos
    \item Editor de texto (\textit{"Find in files"})
  \end{itemize}
\end{frame}

\begin{frame}{Introdução ao Rails e Desenvolvimento Web}
  \huge
  \includegraphics[width=\textwidth]{images/text_editors.png}
\end{frame}

\begin{frame}{Introdução ao Rails e Desenvolvimento Web}
  \huge
  \textbf{Windows 10 / Ubuntu}
  \vfill
  \Large
  \begin{itemize}
    \item Instalar \textbf{editor de texto}
    \item Instalar \textbf{\textit{Bash on Ubuntu on Windows}}
    \item Instalar \textbf{dependências do Ruby} (incl. \textbf{git})
    \item Instalar \textbf{Ruby} (via \textbf{rbenv} ou \textbf{rvm})
    \item Instalar \textbf{bundler}
  \end{itemize}
\end{frame}

\begin{frame}{Introdução ao Rails e Desenvolvimento Web}
  \huge
  \textbf{Geral}
  \vfill
  \begin{itemize}
    \item Instalar \textbf{Rails}
    \item Instalar \textbf{yarn}
  \end{itemize}
\end{frame}

\begin{frame}{Introdução ao Rails e Desenvolvimento Web}
  \huge
  \textbf{Primeira aplicação: \alert{Olá, mundo!}}
  \vfill
  Boa prática: \textbf{pasta de projetos}
\end{frame}

\begin{frame}{Introdução ao Rails e Desenvolvimento Web}
  \huge
  \textbf{Resumo de Bash}
  \vfill
  \begin{center}
    \normalsize
    \begin{tabular}{@{}ll@{}}
      \toprule
      \textbf{Descrição} & \textbf{Comando} \\
      \midrule
      Listar arquivos & \texttt{ls} \\
      Criar diretório & \texttt{mkdir <diretorio>} \\
      Mudar de diretório & \texttt{cd <diretorio>} \\
      Voltar um diretório & \texttt{cd ..} \\
      Mover um arquivo & \texttt{mv <orig> <dest>} \\
      Copiar um arquivo & \texttt{cp <orig> <dest>} \\
      Remover um arquivo & \texttt{rm <arquivo>} \\
      Remover um diretório vazio & \texttt{rmdir <diretorio>} \\
      Remover um diretório não vazio & \texttt{rm -rf <diretorio>} \\
      Exibir o conteúdo de um arquivo & \texttt{cat <arquivo>} \\
      \bottomrule
    \end{tabular}
  \end{center}
\end{frame}

\begin{frame}[fragile]{Introdução ao Rails e Desenvolvimento Web}
  \huge
  \textbf{Criando um novo projeto Rails}
  \vfill
  \Large
  \begin{verbatim}
rails _6.0.0_ new nome_do_projeto
  \end{verbatim}
  \vfill
  \begin{itemize}
    \item Estrutura de pastas
    \item \textbf{Bundler}
  \end{itemize}
\end{frame}

\setbeamertemplate{frame footer}{https://rubygems.org/}
\begin{frame}{Introdução ao Rails e Desenvolvimento Web}
  \huge
  \includegraphics[width=\textwidth]{images/rubygems.png}
\end{frame}
\setbeamertemplate{frame footer}{}

\begin{frame}[fragile]{Introdução ao Rails e Desenvolvimento Web}
  \huge
  \textbf{Rodando a aplicação}
  \vfill
  \Large
  \begin{verbatim}
rails server
  \end{verbatim}
  \vfill
  \begin{itemize}
    \item Endereço -- \texttt{http://localhost:3000}
    \item [\alert{\faExclamation}] Possível configuração necessária
  \end{itemize}
\end{frame}

\begin{frame}{Introdução ao Rails e Desenvolvimento Web}
  \huge
  \textbf{Exercícios}
  \vfill
  \large
  \begin{enumerate}
    \item Qual a versão do \textbf{Ruby}, de acordo com a página inicial da aplicação? Confirme com \texttt{ruby -v}
    \item Qual a versão do \textbf{Rails}?
  \end{enumerate}
\end{frame}

\begin{frame}{Introdução ao Rails e Desenvolvimento Web}
  \huge
  \textbf{Arquitetura MVC}
  \vfill
  \begin{itemize}
    \item \textbf{M}odelo
    \item \textbf{V}isão
    \item \textbf{C}ontrole
  \end{itemize}
\end{frame}

\begin{frame}{Introdução ao Rails e Desenvolvimento Web}
  \huge
  \textbf{Finalmente... \alert{Olá, mundo!}}
\end{frame}

\begin{frame}{Introdução ao Rails e Desenvolvimento Web}
  \huge
  \textbf{Exercícios}
  \vfill
  \Large
  \begin{enumerate}
    \item Exiba a mensagem \textbf{"Olá, pessoal!"} ao invés de "Olá, mundo!" na tela inicial
    \item Crie uma segunda ação \textbf{\texttt{tchau}} que exiba o texto \textbf{"Tchau, mundo!"} e modifique o arquivo \textbf{\texttt{routes.rb}} de forma que esse texto seja exibido na página inicial
  \end{enumerate}
\end{frame}

\begin{frame}{Introdução ao Rails e Desenvolvimento Web}
  \huge
  \includegraphics[width=\textwidth]{images/git.png}
\end{frame}

\begin{frame}[fragile]{Introdução ao Rails e Desenvolvimento Web}
  \huge
  \textbf{Configurações inciais}
  \vfill
  \large
  \begin{verbatim}
git config --global user.name <nome>

git config --global user.email <e-mail>

git config --global
  credential.helper "cache --timeout=86400"
  \end{verbatim}
\end{frame}

\begin{frame}[fragile]{Introdução ao Rails e Desenvolvimento Web}
  \huge
  \textbf{Criando um repositório local}
  \vfill
  \begin{verbatim}
git init
  \end{verbatim}
  \vfill
  \large
  O Rails é tão legal que ele já faz isso pra você \faSmileO
\end{frame}

\begin{frame}[fragile]{Introdução ao Rails e Desenvolvimento Web}
  \huge
  \textbf{Adicionando arquivos para serem salvos}
  \vfill
  \LARGE
  \begin{lstlisting}
git add <arquivo>

git add -A %*\textnormal{(adiciona TUDO)}*)
  \end{lstlisting}
\end{frame}

\begin{frame}[fragile]{Introdução ao Rails e Desenvolvimento Web}
  \huge
  \textbf{Checando as mudanças a serem salvas}
  \vfill
  \LARGE
  \begin{lstlisting}
git status
  \end{lstlisting}
\end{frame}

\begin{frame}[fragile]{Introdução ao Rails e Desenvolvimento Web}
  \huge
  \textbf{Salvando as mudanças}
  \vfill
  \LARGE
  \begin{lstlisting}
git commit -m "<mensagem>"
  \end{lstlisting}
\end{frame}

\begin{frame}[fragile]{Introdução ao Rails e Desenvolvimento Web}
  \huge
  \textbf{Histórico de mudanças}
  \vfill
  \LARGE
  \begin{lstlisting}
git log
  \end{lstlisting}
\end{frame}

\begin{frame}{Introdução ao Rails e Desenvolvimento Web}
  \huge
  \includegraphics[width=\textwidth]{images/github.jpg}
\end{frame}

\begin{frame}[fragile]{Introdução ao Rails e Desenvolvimento Web}
  \huge
  \textbf{Usando git para implementar novas funções}
  \vfill
  \large
  \begin{lstlisting}
git checkout -b nova-funcao
git branch

%*\textnormal{(Desenvolve o código e realiza os commits)}*)

git checkout master
git merge nova-funcao
git branch -d nova-funcao

git push %*\textnormal{(Envia para o GitHub)}*)
  \end{lstlisting}
\end{frame}

\end{document}
