\documentclass[10pt]{beamer}

\usepackage[utf8]{inputenc}
\usepackage[T1]{fontenc}
\usepackage{alphalph}
\usepackage[shortlabels]{enumitem}
\usepackage{caption}
\usepackage{amssymb}
\usepackage{amsmath}
\usepackage{fontawesome}
\usepackage{geometry}
\usepackage[brazil]{babel}
\usepackage{listings}
\usepackage{color}
\usepackage{xcolor}
\usepackage{plex-mono}
\usepackage[sfdefault]{plex-sans}
\usepackage{booktabs}
\usepackage{graphicx}
\usepackage{ragged2e}
\usepackage[none]{hyphenat}
\usepackage[outputdir={../pdf/build_pdflatex}]{minted}
\usepackage{multicol}
\usepackage{pgfplots}
\usepackage{ulem}
\usepackage{tikz}
\usepackage{hyperref}
\usepackage{textcomp}
\usepackage{blkarray}
\usepackage{titling}

\renewcommand{\thefootnote}{\fnsymbol{footnote}}
\renewcommand{\MintedPygmentize}{/home/thiago/.local/bin/pygmentize}

\definecolor{light-gray}{gray}{0.95}
\renewcommand\lstlistingname{Código}
\DeclareCaptionFormat{listing} {
  \parbox{\textwidth}{\hspace{-0.2cm}#1#2#3}
}
\DeclareCaptionFont{black}{\color{black}}
\captionsetup[lstlisting]{
  format=listing,
  labelfont=black,
  textfont=black,
  singlelinecheck=true,
  margin=0pt,
  font={tt,footnotesize,bf}
}

\lstset{
  basicstyle=\footnotesize\ttfamily,
  escapeinside={\%*}{*)},
  mathescape=true,
  showspaces=false,
  showtabs=false,
  showstringspaces=false,%
  rulesepcolor=\color{black},
  frame=shadowbox,
  upquote=true,
  literate=
  {á}{{\'a}}1 {é}{{\'e}}1 {í}{{\'i}}1 {ó}{{\'o}}1 {ú}{{\'u}}1
  {Á}{{\'A}}1 {É}{{\'E}}1 {Í}{{\'I}}1 {Ó}{{\'O}}1 {Ú}{{\'U}}1
  {à}{{\`a}}1 {è}{{\`e}}1 {ì}{{\`i}}1 {ò}{{\`o}}1 {ù}{{\`u}}1
  {À}{{\`A}}1 {È}{{\'E}}1 {Ì}{{\`I}}1 {Ò}{{\`O}}1 {Ù}{{\`U}}1
  {ä}{{\"a}}1 {ë}{{\"e}}1 {ï}{{\"i}}1 {ö}{{\"o}}1 {ü}{{\"u}}1
  {Ä}{{\"A}}1 {Ë}{{\"E}}1 {Ï}{{\"I}}1 {Ö}{{\"O}}1 {Ü}{{\"U}}1
  {â}{{\^a}}1 {ê}{{\^e}}1 {î}{{\^i}}1 {ô}{{\^o}}1 {û}{{\^u}}1
  {Â}{{\^A}}1 {Ê}{{\^E}}1 {Î}{{\^I}}1 {Ô}{{\^O}}1 {Û}{{\^U}}1
  {Ã}{{\~A}}1 {ã}{{\~a}}1 {Õ}{{\~O}}1 {õ}{{\~o}}1
  {œ}{{\oe}}1 {Œ}{{\OE}}1 {æ}{{\ae}}1 {Æ}{{\AE}}1 {ß}{{\ss}}1
  {ű}{{\H{u}}}1 {Ű}{{\H{U}}}1 {ő}{{\H{o}}}1 {Ő}{{\H{O}}}1
  {ç}{{\c c}}1 {Ç}{{\c C}}1 {ø}{{\o}}1 {å}{{\r a}}1 {Å}{{\r A}}1
  {€}{{\euro}}1 {£}{{\pounds}}1 {«}{{\guillemotleft}}1
  {»}{{\guillemotright}}1 {ñ}{{\~n}}1 {Ñ}{{\~N}}1 {¿}{{?`}}1
}

\setminted{
  fontsize=\footnotesize,
  style=bw,
  frame=single,
  labelposition=topline,
}

\pretitle{\begin{center}\normalsize\bfseries\MakeUppercase{Desenvolvimento de Software para a Web II -- }\MakeUppercase}
\author{\normalsize Prof. Thiago Cavalcante}
\date{\vspace{-4ex}}

\geometry{
  top=0.0cm,
  left=1.0cm,
  right=1.0cm,
  bottom=1.0cm
}

\setlength\columnsep{30pt}

\usetikzlibrary{arrows}

\makeatletter
\AddEnumerateCounter{\PaddingUp}{\two@digits}{A00}
\AddEnumerateCounter{\PaddingDown}{\two@digits}{A00}
\newcommand\PaddingUp[1]{\expandafter\two@digits\csname c@#1\endcsname}
\newcommand\PaddingDown[1]{\PaddingUp{#1}\addtocounter{#1}{-2}}
\makeatother

\makeatletter
\def\enumalphalphcnt#1{\expandafter\@enumalphalphcnt\csname c@#1\endcsname}
\def\@enumalphalphcnt#1{\AlphAlph{#1}}
\makeatother
\AddEnumerateCounter{\enumalphalphcnt}{\@enumalphalphcnt}{aa}

\pagenumbering{gobble}

\title{Desenvolvimento de Software para a Web II}
\author{Thiago Cavalcante  -- thiago.kun@gmail.com}
\institute{Universidade Federal de Alagoas -- UFAL \\ Campus Arapiraca \\ Unidade de Ensino de Penedo}
\titlegraphic{\hfill\includegraphics[height=1.5cm]{images/brasao-ufal.eps}}

\subtitle{Aula 4}
\date{14 de novembro de 2019}

\begin{document}

\maketitle

\begin{frame}{Introdução ao Rails e Desenvolvimento Web}
  \HUGE
  \centering
  \textbf{Deployment}
\end{frame}

\begin{frame}{Introdução ao Rails e Desenvolvimento Web}
  \centering
  \includegraphics[width=\textwidth]{images/lion_king.jpg}
\end{frame}

\begin{frame}{Introdução ao Rails e Desenvolvimento Web}
  \Large
  \begin{itemize}
    \item Criar um servidor (\textit{hosting})
    \item Instalar o Ruby
    \item Configurar um servidor \textit{web} (\textit{HTTP requests})
    \item Criar um banco de dados
    \item Fazer o \textit{upload} do código
  \end{itemize}
  \vfill
  \large
  Melhor fazer o \textit{deployment} cedo e com frequência, para evitar problemas de integração com o servidor
\end{frame}

\begin{frame}{Introdução ao Rails e Desenvolvimento Web}
  \centering
  \includegraphics[width=\textwidth]{images/hosting_services.png}
\end{frame}

\begin{frame}{Introdução ao Rails e Desenvolvimento Web}
  \centering
  \includegraphics[width=\textwidth]{images/rails_heroku.png}

  \raggedright
  O \textit{deployment} é \textbf{muito simples} se o seu código estiver em um repositório do \textbf{git}
\end{frame}

\begin{frame}{Introdução ao Rails e Desenvolvimento Web}
  \Large
  \begin{itemize}
    \item Configurar o banco de dados
    \item Instalar a interface de linha de comando do Heroku
    \item Fazer login pela interface
    \item Criar o app
    \item Enviar o código
  \end{itemize}
\end{frame}

\begin{frame}{Introdução ao Rails e Desenvolvimento Web}
  \huge
  \textbf{Banco de dados}
  \vfill
  \begin{columns}
    \begin{column}{0.29\textwidth}
      \centering
      \includegraphics[width=\textwidth]{images/postgresql.png}
    \end{column}
    \begin{column}{0.69\textwidth}
      \large
      \begin{itemize}
        \item Adicionar \textbf{gem} do PostgreSQL ao projeto (\textbf{Gemfile})
        \item Atualizar a instalação com o \textbf{\texttt{bundler}}
        \item Salvar a modifcação no \textbf{git}
      \end{itemize}
    \end{column}
  \end{columns}
\end{frame}

\begin{frame}{Introdução ao Rails e Desenvolvimento Web}
  \huge
  \textbf{Interface de linha de comando}
  \vfill
  \large
  \begin{itemize}
    \item \texttt{\footnotesize curl https://cli-assets.heroku.com/install-ubuntu.sh | sh}
    \item \texttt{heroku -v}
    \item \texttt{heroku login -{}-interactive}
    \item \texttt{heroku create}
  \end{itemize}

\end{frame}

\begin{frame}{Introdução ao Rails e Desenvolvimento Web}
  \huge
  \textbf{O comando \texttt{create} retorna duas URLs}

  \vfill
  \Large
  \begin{itemize}
    \item Uma contém o \textbf{site}, para ser acessado pelos usuários, no formato https://nome\_aleatorio.herokuapp.com/
    \item A outra contém o \textbf{repositório git} que será usado pelo Heroku para o \textit{deployment}
  \end{itemize}
\end{frame}

\begin{frame}{Introdução ao Rails e Desenvolvimento Web}
  \huge
  \textbf{Passos para fazer o \textbf{deployment} no Heroku}
  \vfill
  \begin{enumerate}
    \item \texttt{git push heroku master}
  \end{enumerate}
\end{frame}

\begin{frame}{Introdução ao Rails e Desenvolvimento Web}
  \huge
  \textbf{Para ver o site você pode acessar o endereço ou executar o comando \texttt{heroku open}}
\end{frame}

\begin{frame}{Introdução ao Rails e Desenvolvimento Web}
  \huge
  \textbf{Exercícios}
  \vfill
  \large
  \begin{enumerate}
    \item Faça o \textit{deployment} de um site que exibe "Olá, pessoal!" ao invés de "Olá, mundo!"
    \item Faça o \textit{deployment} de um site que executa a ação \texttt{tchau} no controlador
    \item Execute o comando \texttt{heroku help} para ver uma lista de comandos do \texttt{heroku}. Qual deles exibe os logs da execução de um aplicativo?
    \item Utilize o comando da questão anterior para descobrir o último evento na atividade do seu aplicativo
  \end{enumerate}
\end{frame}

\begin{frame}{Introdução ao Rails e Desenvolvimento Web}
  \huge
  \textbf{O que aprendemos até agora}
  \vfill
  \large
  \begin{itemize}
    \item \textbf{O que é} Rails
    \item \textbf{Instalar} o Rails
    \item Criar um \textbf{novo app} e \textbf{executá-lo}
    \item Criar \textbf{ações} no controlador
    \item Adicionar \textbf{rotas} para as ações
    \item Salvar o projeto no \textbf{git}
    \item Fazer o \textit{deployment} do projeto no \textbf{Heroku}
  \end{itemize}
\end{frame}

\begin{frame}{O poder do Rails}
  \HUGE
  \textbf{Segundo app usando \textit{scaffold generators}}
\end{frame}

\begin{frame}{O poder do Rails}
  \LARGE
  \textbf{App: users e microposts ($\approx$ Twitter)}
\end{frame}

\end{document}
